
\section{Setup Postgres}

You can setup a valid PostgreSQL environment including PostGIS by
following the two steps:

\begin{enumerate}
\def\labelenumi{\arabic{enumi}.}
\item
  Visit the \href{http://www.postgresql.org/}{PostgreSQL website} and
  download the current version. (e.g. \textbf{9.3})
\item
  Visit the \href{http://postgis.net/}{PostGIS website} and download the
  current version. (e.g. \textbf{2.1})
\end{enumerate}

If you are running a Debian system, we recommend installing a
precompiled version. In that case, please follow the
\href{https://wiki.postgresql.org/wiki/Apt}{official instructions} to do
so. As soon as you registered the postgres related APT pgdg repository,
you can run the following command:
\texttt{\# apt-get install postgresql-server-dev-9.3 postgresql-9.3-postgis-2.1}

If you are running a Windows system and you are asked to whether or not
insall plugins, during the installation process; feel free to install
the delivered PostGIS plugin. It may not be the newest version, but it
will suffice.

\subsection{Importing the Database}\label{importing-the-database}

\begin{enumerate}
\def\labelenumi{\arabic{enumi}.}
\item
  To import the database, you need to first create a database:
  \texttt{\$ createdb spatial\_election -{}-encoding UNICODE -{}-template=template0    \$ psql spatial\_election    \textgreater{} CREATE EXTENSION postgis;    \textgreater{} CREATE EXTENSION postgis\_topology;    \textgreater{} \textbackslash{}q}
\item
  Then you need to restore the database dump file to your database. You
  need to dowmload the
  \href{https://github.com/a-d/spatial.election/raw/master/spatial.election.data/spatial_db.backup}{spatial\_db.backup
  file}. \texttt{pg\_restore -d spatial\_election spatial\_db.backup}
\item
  Create a role with password according to
  \href{https://github.com/a-d/spatial.election/blob/master/spatial.election.database/src/main/resources/hibernate.cfg.xml}{your
  hibernate.cfg.xml}
\end{enumerate}