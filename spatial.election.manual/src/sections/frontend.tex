
\section{Frontend}

The frontend can be described best by giving general information about:

\begin{enumerate}
\def\labelenumi{\arabic{enumi}.}
\itemsep1pt\parskip0pt\parsep0pt
\item
  Structuring the page
\item
  Building the map
\item
  Getting the data
\end{enumerate}

\subsection{Structuring the page}\label{structuring-the-page}

As mentioned in the Frameworks Section
we used the Twitter Bootstrap Framework to desgin our application.
Though responsive design is not a priority, the framework does provide
easy to use abstractions to structure a page, plus makes it easy to
stick to a coherent design.

By providing HTML tags with predefined class attributes Bootstrap
enables its user to easily enhance his/her application. The following
snippet divides the page horizontally in two sections, one a third of
the total width, the other two thirds.


\begin{lstlisting}[language=HTML]
<div class="row">
	<div class="col-md-4">
        .....
        </div>
	<div class="col-md-8">
        ....
        </div>
</div>
\end{lstlisting}



\subsection{Building the map}\label{building-the-map}

Within the main div tag of the page the map is displayed.



\begin{lstlisting}[language=HTML]
<div class="panel-body" style="padding:0px">
    <div id="mysvg"></div>
</div>
\end{lstlisting}


As described elsewhere the data to be displayed is loaded on startup
with a call to the REST API provided by our backend.

First a \emph{svg} element and a projection are created.

\begin{lstlisting}
var vis = d3.select("#mysvg").append("svg:svg").attr("width", width).attr("height", height); 
var projection = d3.geo.mercator().center([10.45, 51.30]).scale(2500).translate([width/2,height/2]);
\end{lstlisting}


Later all counties are added to this element. Additionally, a zoom
handler (displayed below) and a click handler are added to look closer
at a county and to display the extrapolated data in the column next to
the map. All counties are colored according to the party who won the
majority of the vote.


\begin{lstlisting}
var zooming = false;
var zoom = d3.behavior.zoom().scaleExtent([1, 8])
  .on("zoom", function() {
    if(!zooming) {
        doclick = false;
        zooming = true;
        var scale = d3.event.scale;
        var trans = scale == 1 ? "0.0, 0.0" : d3.event.translate;
        g.attr("transform", "translate(" + trans + ") scale(" + scale + ")");
        features.attr("stroke-width", strokeWidth(scale) + "");
        zooming = false;
    }
});
var g = vis.append("g").call(zoom);
\end{lstlisting}


\subsubsection{Getting the data}\label{getting-the-data}

The data is provided by calls to the RESTful API of the application. For
each class of entities we need we defined a \emph{factory} encapsulating
a \emph{resource} object, a service from the
\href{http://docs.angularjs.org/api/ngResource}{\emph{ngResource}}
module as provided by AngularJS. Through the object returned by the
factory one can conveniently execute \emph{get}, \emph{post}, \emph{put}
and \emph{delete} requests. For our use case \emph{get} is normally
enough. An example implementation is shown below.



\begin{lstlisting}
angular.module('myApp.services', ['ngResource'])
.factory('Counties', [ '$resource',
  function($resource) {
	var baseurl = 'backend/county/votes/';
	return $resource(baseurl + ':level/', {
		id : '@level'
	},
    {
        'get': {
            method: 'GET',
            transformResponse: function (data) {
            	var dat = angular.fromJson(data);
            	return dat.list ? dat.list : dat;
            },
            isArray: true
        }
    });
}])
\end{lstlisting}

A \emph{get} request is dispatched as follows. First the factory is
injected into the controller which is using it by passing it as an
argument, then a simple function invocation is made.



\begin{lstlisting}
// simplified example
angular.module('myApp.controllers', [])
.controller('mapCtrl', function mapCtrl($scope, Counties, ...) {
	Counties.get({'level' : 2}).$promise.then(function(counties) {
		displayWahlkreise(counties);
	});
});
\end{lstlisting}


