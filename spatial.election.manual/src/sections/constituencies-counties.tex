
\section{Constituencies and Counties}

\begin{enumerate}
\def\labelenumi{\arabic{enumi}.}
\itemsep1pt\parskip0pt\parsep0pt
\item
  About Wahlkreise and Landkreise
\item
  Configuration and Visualization in QGIS Desktop
\item
  Setup for Postgres (+Postgis)
\item
  Derivation of \emph{County} \textless{}-\textgreater{}
  \emph{Constituency}
\end{enumerate}

\subsection{1. About Wahlkreise and
Landkreise}\label{about-wahlkreise-and-landkreise}

\emph{{[}Description{]}}

\subsection{2. Configuration and Visualization in QGIS
Desktop}\label{configuration-and-visualization-in-qgis-desktop}

\subsubsection{Requirements}\label{requirements}

Available Layers *
\href{http://www.bundeswahlleiter.de/en/bundestagswahlen/BTW_BUND_13/wahlkreiseinteilung/kartographische_darstellung.html}{Wahlkreise}
* Download:
\href{http://www.bundeswahlleiter.de/de/bundestagswahlen/BTW_BUND_13/wahlkreiseinteilung/wahlkreisgeometrie/Geometrie_Wahlkreise_18DBT_VG1000_ETRS89.zip}{SHP}
* Projection: \textbf{EPSG:3044 - ETRS89 / ETRS-TM32} *
\href{http://www.gadm.org/country}{Landkreise} * Download
\href{http://biogeo.ucdavis.edu/data/gadm2/shp/DEU_adm.zip}{SHP} *
Projection: \textbf{EPSG:4326 - WGS84}

\subsubsection{Visualization}\label{visualization}

\begin{figure}[htbp]
\centering
\includegraphics[width=1.1\textwidth]{../img/K4GjcyV.png}
\caption{WahlkreiseNLandkreise1}
\end{figure}

\subsubsection{Issues}\label{issues}

\begin{figure}[htbp]
\centering
\includegraphics[width=1.1\textwidth]{../img/HdnNLcV.png}
\caption{WahlkreisVLandkreis}
\end{figure}

\begin{enumerate}
\def\labelenumi{\arabic{enumi}.}
\itemsep1pt\parskip0pt\parsep0pt
\item
  \emph{Wahlkreise} and \emph{Landkreise} are not dependent, thus their
  areas are freely intersecting each other.
\item
  \emph{Wahlkreise} and \emph{Landkreise} show a different degree of
  accuracy in position and detail when it comes to border polygons.
\end{enumerate}

\subsection{3. Setup for Postgres
(+Postgis)}\label{setup-for-postgres-postgis}

\begin{enumerate}
\def\labelenumi{\arabic{enumi}.}
\itemsep1pt\parskip0pt\parsep0pt
\item
  Import shapefiles to the database \emph{public} schema.
\end{enumerate}

\begin{itemize}
\itemsep1pt\parskip0pt\parsep0pt
\item
  \emph{Landkreise} -\textgreater{} ``county''
\item
  \emph{Wahlkreise} -\textgreater{} ``constituency''
\end{itemize}

\begin{enumerate}
\def\labelenumi{\arabic{enumi}.}
\setcounter{enumi}{1}
\itemsep1pt\parskip0pt\parsep0pt
\item
  Add two materialized SQL-Views (virtual tables) for intersections /
  dependencies between ``county'' and ``constituency'':
\end{enumerate}

\begin{lstlisting}
-- delete old view
DROP MATERIALIZED VIEW IF EXISTS public.county_intersecting_constituency;

-- create area intersection view
CREATE MATERIALIZED VIEW public.county_intersecting_constituency AS 

 WITH intersections AS (
  SELECT lk.id_3 AS county_id,
	wktrns.wkr_nr AS constituency_id,
	CONCAT(lk.name_3, ' (', lk.name_1, ', ', lk.name_2, ')') AS county_label,
	wktrns.wkr_name AS constituency_label,
	ST_AREA(ST_INTERSECTION(ST_SetSRID(lk.geom,4326), wktrns.shape)::geography) AS area_intersection,
	ST_AREA(ST_SetSRID(lk.geom,4326)::geography) AS area_county,
	ST_AREA(wktrns.shape::geography) AS area_constituency
  FROM
	county lk
	JOIN (select wkr_nr, wkr_name, ST_Transform(ST_SetSRID(geom,3044), 4326) AS shape FROM constituency) AS wktrns
	ON ST_INTERSECTS(ST_SetSRID(lk.geom,4326), wktrns.shape))
	
 SELECT
	county_id,
	constituency_id,
	county_label,
	constituency_label,
	area_intersection,
	area_intersection / ((SELECT SUM(i.area_intersection) FROM intersections i WHERE i.county_id = intersections.county_id)) AS area_quota,
	area_county,
	area_constituency
 FROM intersections;

CREATE UNIQUE INDEX county_intersecting_constituency_unique
  ON public.county_intersecting_constituency (county_id, constituency_id);
\end{lstlisting}

\subsection{4. Derivation of \emph{County} \textless{}-\textgreater{}
\emph{Constituency}}\label{derivation-of-county---constituency}

Take \emph{Berlin} for example:

\begin{lstlisting}
select * from county_intersecting_constituency where county_label like '%Berlin%' order by area_intersection;
\end{lstlisting}

This request results in a join of intersecting counties and
constituencies, with information about the area of the participate
county, constituency and of course the intersection. A normalized
intersection \textbf{``quota''} index is calculated, which defines the
areal influence of a constituency to a county. The sum of all
constituency ``quota'' of a county is 1.

\begin{table}[h]
\scalebox{0.5}{
\begin{tabular}{|c|c|l|l|r|r|r|r|}
\hline\textbf{county\_id} & \textbf{constituency\_id} & \textbf{county\_label} & \textbf{constituency\_label} & \textbf{area\_intersection} & \textbf{area\_quota} & \textbf{area\_constituency} & \textbf{area\_county} \\ \hline 
141	  &	58    		 &	"Berlin"     &	"Oberhavel Havelland II"	 &	  2803665.490080   &	0.00316685814074   &	2472931758.37186   &	885316816.471166 \\ 
141	  &	63    		 &	"Berlin"     &	"Frankfurt (Oder) Oder-{[}...{]}"   &	  5752379.328977   &	0.00649755449466   &	2405885789.94363   &	885316816.471166 \\ 
141	  &	61    		 &	"Berlin"     &	"Potsdam Potsdam-Mittel{[}...{]}"   &	  6948676.937887   &	0.00784882298049   &	 727030312.23901   &	885316816.471166 \\ 
141	  &	62    		 &	"Berlin"     &	"Dahme-Spreewald Teltow{[}...{]}"   &	  8343997.350849   &	0.00942489609776   &	3973353611.91130   &	885316816.471166 \\ 
141	  &	59    		 &	"Berlin"     &	"M\"arkisch-Oderland Barn{[}...{]}"   &	  8982614.974483   &	0.01014624157473   &	2864879802.60181   &	885316816.471166 \\ 
141	  &	83    		 &	"Berlin"     &	"Berlin-Friedrichshain-K{[}...{]}"   &	 28018823.320075   &	0.03164843988671   &	  28018823.32007   &	885316816.471166 \\ 
141	  &	75    		 &	"Berlin"     &	"Berlin-Mitte"			 &	 45474155.779447   &	0.05136497236676   &	  45474155.77944   &	885316816.471166 \\ 
141	  &	82    		 &	"Berlin"     &	"Berlin-Neuk\"olln"	   	 &	 50038162.639312   &	0.05652021015447   &	  51552590.88985   &	885316816.471166 \\ 
141	  &	81    		 &	"Berlin"     &	"Berlin-Tempelhof-Sch\"one{[}...{]}"   &	 50911846.188703   &	0.05750707248545   &	  52138379.69080   &	885316816.471166 \\ 
141	  &	86    		 &	"Berlin"     &	"Berlin-Lichtenberg"		 &	 54387933.428961   &	0.06143345928648   &	  54578186.82576   &	885316816.471166 \\ 
141	  &	85    		 &	"Berlin"     &	"Berlin-Marzahn-Hellersd{[}...{]}"   &	 55587278.018553   &	0.06278816946519   &	  56213685.87850   &	885316816.471166 \\ 
141	  &	80   		 &	"Berlin"     &	"Berlin-Charlottenburg-W{[}...{]}"   &	 64557831.259508   &	0.07292078680438   &	  64557831.25950   &	885316816.471166 \\ 
141	  &	78   		 &	"Berlin"     &	"Berlin-Spandau Charlot{[}...{]}"   &	 70910610.225396   &	0.08009651795193   &	  83642855.89283   &	885316816.471166 \\ 
141	  &	77    		 &	"Berlin"     &	"Berlin-Reinickendorf"		 &	 77263528.269943   &	0.08727240619467   &	  84103703.05926   &	885316816.471166 \\ 
141	  &	76    		 &	"Berlin"     &	"Berlin-Pankow"			 &	 90999372.170865   &	0.10278761984321   &	  95504666.59105   &	885316816.471166 \\ 
141	  &	79    		 &	"Berlin"     &	"Berlin-Steglitz-Zehlend{[}...{]}"   &	102307387.695839   &	0.11556049918550   &	 106561169.57164   &	885316816.471166 \\ 
141	  &	84    		 &	"Berlin"     &	"Berlin-Treptow-K\"openick"	 &	162026255.436742   &	0.18301547308679   &	 167319473.44736   &	885316816.471166 \\  \hline 
             
\end{tabular}
}
\end{table}


\emph{Notice:} Since the map data of constituencies and counties are
inaccurately overlapping a part of \emph{Brandenburg} constituencies
intersect with the county \emph{Berlin}. The sum of all
\textbf{``quota''} from \emph{Brandenburg} constituencies in
\emph{Berlin} county is 0.03708, which means 3.7\% of \emph{Berlin}s
area will be influenced by voting results descending from constituency
of \emph{Brandenburg}.

