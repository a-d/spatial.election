\documentclass[ucs,9pt]{beamer}

% Copyright 2004 by Till Tantau <tantau@users.sourceforge.net>.
%
% In principle, this file can be redistributed and/or modified under
% the terms of the GNU Public License, version 2.
%
% However, this file is supposed to be a template to be modified
% for your own needs. For this reason, if you use this file as a
% template and not specifically distribute it as part of a another
% package/program, I grant the extra permission to freely copy and
% modify this file as you see fit and even to delete this copyright
% notice.
%
% Modified by Tobias G. Pfeiffer <tobias.pfeiffer@math.fu-berlin.de>
% to show usage of some features specific to the FU Berlin template.

\usepackage[utf8x]{inputenc}
\usepackage[english]{babel}
\usepackage{arev,t1enc}
\include{fu-beamer-template}

% beamer options
\titleimage{fu_500}
\title[Spatial Databases -- Projekt]{Visualizing Results of the General Election in Germany 2013  Spatial Databases  Prof. Dr. Agnes Voisard}
\author{Alexander D\"umont \and Martin Liesenberg}
\institute[FU Berlin]{Freie Universität Berlin}
\renewcommand{\footlinetext}{\insertshortinstitute, \insertshorttitle, \insertshortdate}
\usenavigationsymbolstemplate{}
\setbeamertemplate{bibliography item}[text]

% tikz options
\usepackage{tikz-er2}
%https://bitbucket.org/pavel_calado/tikz-er2/raw/da9f9f7f169647cad6d91df7975400b1605ae67a/tikz-er2.sty
\usetikzlibrary{positioning}
\usetikzlibrary{shadows}
\tikzstyle{every entity} = [top color=white, bottom color=blue!30, 
                            draw=blue!50!black!100, drop shadow]
\tikzstyle{every weak entity} = [drop shadow={shadow xshift=.7ex, 
                                 shadow yshift=-.7ex}]
\tikzstyle{every attribute} = [top color=white, bottom color=yellow!20, 
                               draw=yellow, node distance=1cm, drop shadow]
\tikzstyle{every relationship} = [top color=white, bottom color=red!20, 
                                  draw=red!50!black!100, drop shadow]
\tikzstyle{every isa} = [top color=white, bottom color=green!20, 
                         draw=green!50!black!100, drop shadow]

% lstlisting options
\usepackage{listings}
\usepackage{color}

\definecolor{mymauve}{rgb}{0.58,0,0.82}
\definecolor{mygray}{rgb}{0.5,0.5,0.5}

\lstdefinelanguage{JavaScript}{
     keywords={var},
     keywordstyle=\color{purple}\bfseries,
     ndkeywords={d3, geo, mercator, center, scale, translate},
     ndkeywordstyle=\color{mymauve}\slshape,
     identifierstyle=\color{black},
     sensitive=false,
     comment=[l]{//},
     commentstyle=\color{green}\ttfamily,
     stringstyle=\color{red}\ttfamily
  }

\lstset{ %
  backgroundcolor=\color{white},   % choose the background color; you must add \usepackage{color} or \usepackage{xcolor}
  basicstyle=\footnotesize,        % the size of the fonts that are used for the code
  breakatwhitespace=false,         % sets if automatic breaks should only happen at whitespace
  breaklines=true,                 % sets automatic line breaking
  captionpos=b,                    % sets the caption-position to bottom
  extendedchars=true,              % lets you use non-ASCII characters; for 8-bits encodings only, does not work with UTF-8
  frame=single,                    % adds a frame around the code
  keepspaces=true,                 % keeps spaces in text, useful for keeping indentation of code (possibly needs columns=flexible)
  language=JavaScript,             % the language of the code
  numbers=left,                    % where to put the line-numbers; possible values are (none, left, right)
  numbersep=5pt,                   % how far the line-numbers are from the code
  numberstyle=\small\color{mygray}, % the style that is used for the line-numbers
  rulecolor=\color{black},         % if not set, the frame-color may be changed on line-breaks within not-black text (e.g. comments (green here))
  showspaces=false,                % show spaces everywhere adding particular underscores; it overrides 'showstringspaces'
  showstringspaces=false,          % underline spaces within strings only
  showtabs=false,                  % show tabs within strings adding particular underscores
  stepnumber=1,                    % the step between two line-numbers. If it's 1, each line will be numbered
  stringstyle=\color{mymauve},     % string literal style
  tabsize=2,                       % sets default tabsize to 2 spaces
  title=\lstname                   % show the filename of files included with \lstinputlisting; also try caption instead of title
}


\AtBeginSection[]
{
  \begin{frame}<beamer>{Outline}
    \tableofcontents[currentsection,currentsubsection]
  \end{frame}
}

\begin{document}

\begin{frame}[plain]
  \titlepage
\end{frame}

\begin{frame}{Outline}
  \tableofcontents
\end{frame}

\section{Einleitung}
\begin{frame}{Bundestagswahl 2013}
	\begin{figure}[hbtp]
		\centering
		\includegraphics[scale=0.45]{ergebniswahl.png}
		\caption{Ergebnisse der Bundestagswahl 2013}
	\end{figure}
\end{frame}

\begin{frame}{Projekt - Aufgabenstellung}
	\begin{columns}[c] % the "c" option specifies center vertical alignment
  		\begin{column}[T]{0.4\textwidth} % each column can also be its own environment
			\textbf{Unser Ziel:}
			\begin{quote}
				``Visuelles Werkzeug zur Analyse und Auswertung von Wahlergebnissen in Bezug auf Einwohner- und Standortdaten."
			\end{quote}					
		\end{column}
		\begin{column}[T]{0.45\textwidth} % each column can also be its own environment
			\begin{figure}[hbtp]
				\centering
				\includegraphics[scale=0.3]{ergebniswahl.png}
				\caption{Screenshot unserer Applikation}
			\end{figure}
  		\end{column}
  	\end{columns}
\end{frame}

\section{Projekt - Umsetzung}
\begin{frame}{Technologien}
	\begin{columns}[c] % the "c" option specifies center vertical alignment
  		\begin{column}[T]{0.49\textwidth} % each column can also be its own environment
			\textbf{Technologien}
			\begin{itemize}
				\item \href{http://www.java.com/en/}{Java} \includegraphics[scale=0.08]{javalogo.png}
				\item \href{https://jersey.java.net/}{Jersey} \includegraphics[scale=0.5]{jerseylogo.png} 
				\item \href{http://www.postgresql.org/}{PostgreSQL} \includegraphics[scale=0.08]{postgresqllogo.png}
				\item \href{http://postgis.net/}{PostGIS} \includegraphics[scale=0.08]{postgislogo.png}
			\end{itemize}
		\end{column}
		\begin{column}[T]{0.49\textwidth} % each column can also be its own environment
			\textbf{Frameworks}
			\begin{itemize}
				\item \href{http://www.hibernatespatial.org/}{Hibernate Spatial} \includegraphics[scale=0.08]{hibernatelogo.png}
				\item \href{http://angularjs.org/}{AngularJS} \includegraphics[scale=0.05]{angularjslogo.png}
				\item \href{http://d3js.org/}{D3.js} \includegraphics[scale=0.06]{d3jslogo.jpg}
				\item \href{http://getbootstrap.com/}{Twitter Bootstrap} \includegraphics[scale=0.08]{bootstraplogo.png}
			\end{itemize}
  		\end{column}
  	\end{columns}
\end{frame}

\begin{frame}{Datenmodell}
	\begin{center}
	\scalebox{.5}{
		\begin{tikzpicture}[node distance=1.5cm, every edge/.style={link}]

			\node[entity] (elec) {Election};
			\node[attribute] (elecId) [above=of elec] {\key{electionID}} edge (elec);
			\node[attribute] (elecDate) [above left=of elec] {date} edge (elec);

			\node[relationship] (hasE) [right=1cm of elec] {Has} edge node[auto,swap] {1} (elec);

			\node[entity] [right=1cm of hasE] (electionRes) {ElectionResult} edge node[auto,swap] {1...*} (hasE);
			\node[attribute] (electionResEId) [above=of electionRes] {\key{electionID}} edge (electionRes);
			\node[attribute] (electionResPId) [above right=of electionRes] {\key{partyID}} edge (electionRes);
			\node[attribute] (electionResPVotes) [above left=of electionRes] {Primary Votes} edge (electionRes);
			\node[attribute] (electionResSVotes) [below left=of electionRes] {Secondary Votes} edge (electionRes);
			
			\node[relationship] (hasP) [right=1cm of electionRes] {Has} edge node[auto,swap] {0...*} (electionRes);
			
			\node[entity] [right=1cm of hasP] (party) {Party} edge node[auto,swap] {1} (hasP);
			\node[attribute] (partyID) [above=of party] {\key{partyID}} edge (party);
			\node[attribute] (partyName) [above right=of party] {Name} edge (party);
			
			\node[relationship] (hasC) [below=1cm of electionRes] {Has} edge node[auto,swap] {0...*} (electionRes);			
			
			\node[entity] [below=1cm of hasC] (cons) {Constituency} edge node[auto,swap] {1} (hasC);
			\node[attribute] (consID) [left=of cons] {\key{constituencyID}} edge (cons);
			\node[attribute] (consName) [below left=of cons] {Name} edge (cons);
			\node[attribute] (consGeom) [below right=of cons] {Geom} edge (cons);
			
			\node[relationship] (hasCD) [right=1cm of cons] {Has} edge node[auto,swap] {1} (cons);
			
			\node[entity] [right=1cm of hasCD] (consData) {ConstituencyData} edge node[auto,swap] {1...*} (hasCD);
			\node[attribute] (consDataKey) [above=of consData] {Key} edge (consData);
			\node[attribute] (consDataValue) [above right=of consData] {Value} edge (consData);
			
			\node[relationship] (hasCo) [below=1cm of cons] {Has} edge node[auto,swap] {0...*} (cons);
			
			\node[entity] [below=1cm of hasCo] (county) {County} edge node[auto,swap] {1} (hasCo);
			\node[attribute] (countyID) [left=of county] {\key{countyID}} edge (county);
			\node[attribute] (countyName) [below left=of county] {Name} edge (county);
			\node[attribute] (countyGeom) [below right=of county] {Geom} edge (county);
			
			\node[relationship] (hasCoD) [right=1cm of county] {Has} edge node[auto,swap] {1} (county);
			
			\node[entity] [right=1cm of hasCoD] (countyData) {CountyData} edge node[auto,swap] {1...*} (hasCoD);
			\node[attribute] (countyDataKey) [above=of countyData] {Key} edge (countyData);
			\node[attribute] (countyDataValue) [above right=of countyData] {Value} edge (countyData);
			
		\end{tikzpicture}
	}
	\end{center}
\end{frame}

\begin{frame}{Architektur}
	\begin{figure}[hbtp]
		\centering
		\includegraphics[scale=0.35]{ArchitekturSpatialDB.png}
		\caption{Architektur der Anwendung}
	\end{figure}
\end{frame}

\begin{frame}{Designentscheidungen}
	\begin{figure}[hbtp]
		\centering
		\includegraphics[scale=0.5]{ccissue.png}
		\caption{Wahl- und Landkreisüberlappung}
	\end{figure}
\end{frame}

\begin{frame}[fragile]{Beispielcode}
 	\begin{lstlisting}
		var width = 800, height = 600; // size of map
		var projection = d3.geo.mercator() // type of projection
        	.center([10.45, 51.16]) // center of Germany
        	.scale(2500)
        	.translate([width/2,height/2]);

		var coordinateWGS84 = [13.38, 52.52]; // Berlin
		var coordinateCartesian = projection(coordinateWGS84);

		=> coordinateCartesian = [377.85, 153.95]
  \end{lstlisting}
\end{frame}

\section{Demo}
\begin{frame}{Demo}
	\begin{center}
		\huge Demo	
	\end{center}
\end{frame}


\end{document}
